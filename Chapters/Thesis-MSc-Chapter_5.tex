% #############################################################################
% This is Chapter 5
% !TEX root = ../main.tex
% #############################################################################
% Change the Name of the Chapter i the following line
\fancychapter{Sterio System}
\cleardoublepage
% The following line allows to ref this chapter
\label{chap:steriosystem}

To understand the tasks that our platform must fulfill, the first steps we have taken were an investigation and analysis of the currently available music streaming platforms and terrestrial radio stations, identifying the strengths, weaknesses, and opportunities of each. At the same time, we have conducted a study on the available literature that addresses theses mediums and the vital concept of interactive radio. Further, we have overseen a thorough user research study by conducting a survey, a diary study, and interviews, and, most importantly, by applying the speed dating method. As we're applying user-centered design and human-computer interaction principles and methodologies, our users must be involved in the development of the project from the very early stages. This will maximize the quality of the user experience of the solution, and the earlier the user is involved, the less repair work needs to be done at the final stages of the project's life cycles.~\cite{Courage2005}

After the presented research, we can identify our opportunity and act upon it. As such, in the first stage, we need to determine our requirements and accordingly plan the features and tasks that are going to be made available on the platform to fulfill our users' desires. The gathered datasets from sections ~\ref{chap:relatedwork}, ~\ref{chap:userresearch}, and ~\ref{sec:speeddating} provided pivotal information that helped fulfill this task. Then, after we've outlined the goals that our platform must satisfy, we can outset the development of a functional prototype with a working feature set and near ready for general-purpose usage.

In this section, we explain in greater detail the development process that led us to the final Sterio platform. We begin by outlining the requirements and goals that our solution must fulfill. Afterward, we discuss and examine the adopted technologies and services, as well as the overall architecture of the system. Finally, we present a complete overhaul of the crafted features by describing the methods, technical facets, and reasoning behind all components of the application.


% #############################################################################
\section{Requirements} 

Taking into account all the conducted research regarding previous work, and by identifying and understanding our users' needs, we were able to identify a concrete set of features that we expect our solution to tackle. These features can be described as followed:

\begin{itemize}
	\item Creation of personalized radio stations, allowing users to select their desired audio content (by songs, albums, artists, playlists or others) using an on-demand music streaming service, or even add to the station other audio media content such as podcasts or audiobooks;
	\item A 'virtual radio host' based on text-to-speech technology is attributed to a given station, allowing content to be delivered in the periphery during that session (news, weather, traffic, social feeds, information about friends and family, and other types of readable information);
	\item A high level of customization of such radio stations and of its content must be available, allowing users to choose how often they would like to listen to each sort of content, the specific topics or themes of each audible content, the voice of the 'virtual radio-host' from the selection of the available text-to-speech voices, among other functionalities;
	\item The 'virtual radio host' mimics as best as possible a 'real' radio host, promoting interaction, human connection, and empathy between the listeners and their ‘own radio host’. Plus, audible divisors and elements, as well as other radio-familiar components are introduced along the session, so that these personal radio stations are as natural as possible, reassembling a 'real' radio station;
	\item A high level of shareability of the created radio stations, social/informative content, and other elements, allowing a simultaneous listening experience of radio stations among the platform's users, reproducing the same community feeling as traditional terrestrial radio, while at the same time indulging audio listeners in a social-network like atmosphere.
\end{itemize}

In the end, a general-purpose platform will emerge that creates a novel listening experience by merging the best functionalities of both music streaming services and traditional terrestrial radio in a personalized, integrated and social experience that may be shared with users' friends and family.


% #############################################################################


\section{Architecture}

The Sterio platform was developed following a layered architecture which not only supports the incremental development of systems but provides a changeable structure so that an equivalent layer can replace another one. Moreover, when a given layer is changed or updated, only its adjacent layer is affected. Furthermore, every layer of the Sterio system can be used individually with other similar applications or can be easily changed without compromising the other layers. 

The three main layers that compose our system are the Presentation, the Business, and the Database Layer, represented in Figure CENAS. In the following subsections, we explain in greater detail the role of each layer, as well as the reasoning and advantages of the used frameworks and technologies. 

\subsection{Database Layer}

The Database Layer is responsible for managing and storing all the data that it is used in the Sterio system. It receives information entered by the application's users and answers accordingly with the requested information from the Business Layer. 

The first development step of the platform was the creation of an entity-relationship model so that we can model the database and determine which entities we need based on the medium-fidelity prototype described in section ~\ref{sec:userenactments}. The representation of this model, shown in Figure ~\ref{fig:eadiagram}, will help us visualize and conceptualize the system in the first stage, which will help reduce the development difficulty and discard preliminary oversights.

\begin{figure}[h]
\centering
\includegraphics[width=0.8\textwidth]{./Images/ea.png}
\caption{Entity – Relationship Model of the Sterio system}
\label{fig:eadiagram}
\end{figure}


The implementation of the database was conducted using Google's Cloud Firestore, which is a NoSQL, document-oriented database. Each document contains a set of key-value pairs, being optimized for storing sizable collections of small documents. It is a serverless document database that effortlessly scales to meet any demand, with no maintenance required, which accelerates the development of native cloud applications and lets developers focus their efforts on the most foreground layers of a system.

We chose to use Cloud Firestore due to its lean learning curve, ease-of-use, good performance, reliability, high scalability, and deep integration with other Google services that will also be used in the development of the platform. Furthermore, by using this technology, the system is prepared to be easily customized and to receive new data if the project has any changes in the way we approach some of its points.


\subsection{Business Layer}

\subsection{Presentation Layer}

The third layer is the Presentation Layer, which is responsible for the interaction between the user and the system. This layer will interact with the business layer through calls to the REST API.

Based on the preliminary user research presented in section ~\ref{chap:userresearch}, and corroborating with the data shown in Figure ~\ref{chart:devices}, most users listen to music streaming services on their smartphone. Furthermore, as we want our platform to be easily accessible on the go, we focused our efforts in analyzing the most popular mobile development frameworks to develop our platform on.

\begin{figure}
	\centering
	\caption{Share of internet users who have used a music streaming services in the last month worldwide in 2nd quarter 2017, by device (Statista / GlobalWebIndex)}
	\label{chart:devices}
	\begin{bchart}[step=10,max=45,unit=\%,width=0.8\textwidth]
        \bcbar[text=Smartphone (Mobile)]{39}
            \smallskip
        \bcbar[text=PC / Laptop]{30}
            \smallskip
        \bcbar[text=Tablet]{8}
    \end{bchart}
\end{figure}


We chose to develop the Sterio platform using Flutter, which is an UI toolkit for building natively compiled applications for mobile, web, and desktop from a single codebase. Flutter apps are written in the Dart programming language and make use of many of the language's more advanced features.

In the context of our project, Flutter has some key advantages over other technologies. To start, although it has been built as a mobile-first toolkit in the first stage, Flutter is now a cross-platform development tool that allows the development of mobile and desktop apps without compounding changes to the codebase. This ensures that our platform renders well on a variety of devices and windows or screen sizes, without limiting our endeavors. Secondly, in comparison with other mobile frameworks, Flutter reduces the code development time by a wide margin. In a large and complex project such as ours, this is a crucial advantage that will lead us to a robust final product without the need for allocating umpteen resources. Finally, Flutter offers a variety of advanced tools that allow us to achieve a great user experience and interface design, which will help us achieve our goals.


\section{Functional Prototype}
