% #############################################################################
% This is Chapter 7
% !TEX root = ../main.tex
% #############################################################################
% Change the Name of the Chapter i the following line
\fancychapter{Conclusion}
%\cleardoublepage
% The following line allows to ref this chapter
\label{chap:conclusion}

In an age where on-demand streaming services are the preferred way for users to consume audio media content, the human connection that makes these experiences so enjoyable in the first place is lacking more than ever. Although they have full control over what they listen, users tend to find the music streaming listening experience quite dull, tedious, and repetitive after long periods. Conversely, traditional mediums, such as terrestrial radio, are still thriving, in part due to its human element that is lost while using streaming services. Moreover, traditional radio stations keep their listeners 'connected to the world' through the dissemination of information, such as news, traffic reports, and weather information. 

While both media services offer listeners a distinct set of value propositions, efforts to combine the ’best of both worlds’ have been few and far between. Towards this objective, we investigated how we can develop a platform that aims to best represent audio media consumers’ music streaming and traditional terrestrial radio habits into an integrated and personalized experience, that may be shared within small networks of friends and family.

We started by studying the currently available mediums and the concept of interactive radio. Next, we conducted preliminary user research activities that gave us an insight into users’ listening habits and desires. To understand how these habits can be constituted in a new platform that aims to create a novel listening experience while applying the interactive radio approach, we used the speed dating methodology, which validated users' needs, reduced the design dimensions of the platform, and generated a medium-fidelity prototype that was used as the foundation for the development of the platform. 

Finally, we present our value proposition, which consists of the Sterio platform, that was developed with a sturdy focus on the user. The system allows the creation of personalized radio stations that allow users to select their desired music using a streaming service or other audio media content such as podcasts or audiobooks. A high level of shareability and sociability is offered, allowing a simultaneous listening experience of radio stations among the platform's users, reproducing the same community feeling as traditional terrestrial radio, while at the same time indulging audio listeners in a social-network like atmosphere.

To each station, a 'virtual radio host' based on text-to-speech technology is attributed, allowing content to be delivered in the periphery during that session. This host mimics as best as possible a 'real' radio host, promoting interaction, human connection, and empathy between the listeners and their ‘own radio host’. Plus, audible divisors and elements, as well as other radio-familiar components are introduced along with the session, so that these personal radio stations reassemble a 'real' radio station.

After describing in detail the final crafted solution, as well as our efforts, technologies, and methodologies used in the context of the development of the platform, we presented an in-depth analysis of our evaluation methodology and its results. By interpreting them, we concluded that all our objectives were achieved, meaning we successfully created and validated a novel radio-like experience that is at once personal, customizable, and shareable.

% #############################################################################
\section{Future Work}

Considering the current work and the results from usability tests, we consider that some features can be improved:

\begin{itemize}
	\item Enhance the playing algorithm. As previously described, Spotify's \ac{API} provides a limited set of tools to control a user's music playback, which, to bypass such limitations, we had to develop a workaround that applies numerous resources, thus affecting the overall performance of the system.
	\item Integration with extra music streaming services. On the first hand, we designed Sterio to work with Spotify solely, as at the time of writing was the most used in the world, but more and more people are using other streaming services, such as Apple Music or Tidal. This expanded integration would amplify the target audience of the platform by a wide margin.
	\item Alliance with traditional terrestrial radio stations. Users of the Sterio platform want to fully customize and create unique radio-like experiences, and that could be further augmented by incorporating elements from traditional radio stations – such as radio shows, interviews, or even pre-set stations.
	\item Creation of more custom blocks. Although we started with 6 simple information blocks, the core foundation of the developed system is designed so that there is easy expandability in terms of station's blocks.
	\item Portability for the web. The current system only supports the iOS and Android mobile operating systems, but, as we used Flutter as the development framework, it is easy to conceive a web — or even native desktop — platform.
\end{itemize}


