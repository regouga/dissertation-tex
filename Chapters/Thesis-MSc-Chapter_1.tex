% #############################################################################
% This is Chapter 1
% !TEX root = ../main.tex
% #############################################################################
% Change the Name of the Chapter i the following line
\fancychapter{Introduction}
\cleardoublepage
% The following line allows to ref this chapter
\label{chap:intro}
At the start of the millennium, Coats et. al ~\cite{Coats2000} predicted that streaming would become the future of audio media consumption. With respect to music streaming services and corresponding developments around related technologies and the internet, early studies anticipated the stagnation and ultimate demise of traditional media, such as terrestrial radio ~\cite{Ala-Fossi2008}. More recent research, however, appears to contradict these predictions, revealing the sustained popularity of traditional radio broadcasting. ~\cite{DangNguyen2012, Waits2007} Indeed, in large parts of the world, traditional radio remains strong and continues to co-exist alongside newer streaming services, albeit with the important caveat that younger audiences are diminishing ~\cite{Albarran2007}. 

Streaming has rapidly become the standard delivery method for digital entertainment content ~\cite{Swaminathan2013}, with the music industry forming an integral part of this interactive mode of media conveyance. In recent years, platforms such as \textit{Spotify}, \textit{Apple Music}, and \textit{Tidal} have emerged as some of the more predominant and thriving services for on-demand media consumption, offering users new and easier ways to access, listen to, and discover songs, artists ~\cite{Weijters2014} and, more recently, podcasts to match their tastes. Specifically, audio streaming services enable listeners to access and discover an almost limitless selection of content ~\cite{Morris2015}. With their ubiquity and large catalogs of recorded music and podcasts, along with social functions — such as the ability to create and collaborate on playlists, group listening, and shared activity notifications — audio streaming services offer listeners an enticing array of experiences, resulting in the widespread adoption of these services. ~\cite{Mantymaki2015}

Traditional radio, on the other hand, delivers a connection to the outside world through the disclosure of important information in a succinct way. More importantly, and in contrast to music streaming services, it is difficult for radio stations to make their song selection appealing to every listener, which in return makes them get worn-out and tired of tuning in to radio stations. 

Yet, traditional terrestrial radio's popularity has remained very strong in recent years. ~\cite{Albarran2007} This is, in part, due to the human connection this medium provides, and which other modern solutions are taking away ~\cite{Waits2007}. The 'social presence element', described by Short et. al ~\cite{JohnShortEderynWilliams1976} as "the degree to which a particular medium allows communicators to feel other people as being present psychologically", is lacking in music streaming services. The authors state that in conjunction with the lack of nonverbal cues —  which makes the communication quite limited — there is a direct and indirect impact on users’ behavioral intention or actual use of technological platforms, such as music streaming services ~\cite{Wang2014}.

From the beginning of its adoption, terrestrial radio's strengths were ubiquity of access, ease of use, and the local nature of its content, as stated by the North American Broadcasters Association (NABA) ~\footnote{\href{https://nabanet.com/wp-content/uploads/2019/03/NGR-WG-Value-Proposition-of-Radio-in-a-Connected-World-2019-03-15.pdf}{The Value Proposition Of Radio In A Connected World — NABA Next Generation Radio Working Group, 2019}}. Furthermore, according to Priestman et. al ~\cite{Priestman2005}, one of the most compelling reasons for people to listen to it is because of the intimacy of audio — a person listening to the radio is alone with the announcer or artist, even if other people are physically present, and much of the fascination of audio is the imagination it requires on the part of the listener to actively visualize. Waits et. al ~\cite{Waits2007} also states that listening to the radio, though experienced individually, is often a communal act, which sets our relationship to traditional radio to be determined by a certain expectation that it will be authentic and sociable.

Bringing all together, we can conclude that there is a lack of solutions that aim to improve the audio media consumers' experience. Music streaming services are convenient and highly popular because they allow listeners to not only enjoy their favorite songs on demand but also to discover brand new artists that match their music taste. On the downside, they eliminate the human connection that traditional terrestrial radio stations provide, as there isn't someone on the other side of the line interacting with the listener, nor communicating information such as news, weather, or traffic information. Therefore, listeners lose their connection to the outside world while pivoting themselves on music streaming services. To try to improve this experience, we have started by asking ourselves: how can audio media consumers' music streaming and traditional terrestrial radio habits be best represented in an integrated and personalized experience that may be shared within small networks of friends and family?

In this work, we describe our efforts in designing and conceiving a solution, dubbed \textit{Sterio}, that aims to answer our hunt statement, by developing a platform that is user-focused from its inception. Through rapid prototyping and the speed dating method, we explore the design implications for creating and validating such radio-like experiences that are at once personal, customizable, and shareable.

% #############################################################################
\section{Goals}

The objective of this work is to develop a general-purpose platform that creates a novel radio-like listening experience, that aims to be personal, yet personalized and social. In order to achieve this goal, we defined some sub-objectives:

\begin{itemize}
	\item Identify the most and least valued user features of both music streaming services and traditional terrestrial radio;
	\item Study and analyze the currently available platforms and mediums, as well as their most recent augmentations; 
	\item Explore, develop, and reflect on the stature of a set of concepts and prototypes that aim to tailor users' needs and desires into an audio-listening experience of the context of this work;
	\item Design and develop a solid, appealing, consistent, and user-focused functional prototype of a general-purpose platform, that creates a novel and enticing radio-like listening experience;
	\item Evaluate with users the functional prototype, in order to understand what type of experience is created within its users, as well as its usability, viability, and likability.
\end{itemize}



% #############################################################################
\section{Document Structure}

In Chapter ~\ref{chap:relatedwork} we present and discuss related work, focusing on the currently available music streaming services, how traditional terrestrial radio still plays an important role in audio media conveyance, and how the concept of interactive radio can be further augmented. At the end of such a chapter, we define the requirements for our solution. Chapter ~\ref{chap:userresearch} is dedicated to describing the preliminary user research we conducted, and Chapter ~\ref{sec:speeddating} presents the results of applying the speed-dating methodology to our concept. Chapter ~\ref{chap:steriosystem} describes the implemented solution, and Chapter ~\ref{chap:evaluation} shows the evaluation conducted on prototypes and its results. Finally, in Chapter ~\ref{chap:conclusion}, we expose our conclusions on this work and reflect on future work.

