% #############################################################################
% Preamble for Thesis-MSc in English or Portuguese
% Required Packages and commands
% --> Please Choose the MAIN LANGUAGE for the Thesis in package BABEL (below)
% !TEX root = ./main.tex
% #############################################################################
% Thesis-MSc
% Version 2.0, August 2018
% BY: Rui Santos Cruz, rui.s.cruz@tecnico.ulisboa.pt
% #############################################################################
%
% -----------------------------------------------------------------------------
% PACKAGES ucs, utf8x, babel, iflang:
% -----------------------------------------------------------------------------
% The 'ucs' package provides support for using UTF-8 in LaTeX documents. 
% However in most situations it is not required.
\usepackage{ucs}
% The 'utf8x' package contains support for using UTF-8 as input encoding. 
\usepackage[utf8x]{inputenc}
% The 'babel' package may correct some hyphenation issues of LaTeX. 
% Select your MAIN LANGUAGE for the Thesis with the 'main=' option.
\usepackage[main=english,portuguese]{babel}
% The 'iflang' package is used to help determine the language being used. 
\usepackage{iflang}

% -----------------------------------------------------------------------------
% PACKAGE scrbase:
% -----------------------------------------------------------------------------
% The 'scrbase' package is used to help redefining document structure.
\usepackage{scrbase}
% -----------------------------------------------------------------------------
% PACKAGE mathtools, amsmath, amsthm, amssymb, amsfonts, nicefrac:
% -----------------------------------------------------------------------------
% These packages are typically required. 
% Among many other things they add the possibility to put symbols in bold
% by using \boldsymbol (not \mathbf); defines additional fonts and symbols;
% adds the \eqref command for citing equations.
\usepackage{mathtools, amsmath, amsthm, amssymb, amsfonts}
\usepackage{nicefrac}
%
% -----------------------------------------------------------------------------
% PACKAGE tikz:
% -----------------------------------------------------------------------------
% Tikz  for creating graphics programmatically.
\usepackage{tikz}
\usetikzlibrary{shapes.geometric, arrows, positioning}
% -----------------------------------------------------------------------------
% PACKAGES array, booktabs, multirow, colortbl, ctable, spreadtab:
% -----------------------------------------------------------------------------
% These packages are most usefull for advanced tables. 
% 'multirow' allows to join rows throuhg the command \multirow which works
% similarly with the command \multicolumn.
% The 'colortbl' package allows to color the table (foreground and background)
% The 'ctable' package provides commands to easily typeset centered or left or
% right aligned tables.
% The package 'booktabs' provide some additional commands to enhance
% the quality of tables
% The 'longtable' package is only required when tables extend beyond the length
% of one page, which typically does not happen and should be avoided
\usepackage{array}
\usepackage{booktabs}
\usepackage{multirow}
\usepackage{colortbl}
\usepackage{ctable}
\usepackage{spreadtab}
\usepackage{longtable}
\usepackage{pgfplots}
\usepackage{bchart}
\usepackage{hhline}
\usepackage{pdfpages}
\usepackage{epigraph}

%
% -----------------------------------------------------------------------------
% PACKAGES graphicx, subfigure:
% -----------------------------------------------------------------------------
% The package 'graphicx' supports formats PNG and JPG.
% Package 'subfigure' allows to place figures within figures with own caption. 
% For each of the subfigures use the command \subfigure.
\usepackage{graphicx}
\usepackage[hang,small,bf,tight]{subfigure}
%
% -----------------------------------------------------------------------------
% PACKAGE caption:
% -----------------------------------------------------------------------------
% The 'caption' package offers customization of captions in floating 
% environments such figure and table
% \usepackage[hang,small,bf]{caption}
\usepackage[format=hang,labelfont=bf,font=small]{caption} 
% the following customization adds vertical space between caption and the table
\captionsetup[table]{skip=10pt}
%
% -----------------------------------------------------------------------------
% PACKAGE algorithmic, algorithm, algorithm2e:
% -----------------------------------------------------------------------------
% These packages are required if you need to describe an algorithm.
% The preference is for using 'algorithm2e'
%\usepackage{algorithmic}
%\usepackage[chapter]{algorithm}
\usepackage[ruled,vlined,algochapter,norelsize,\languagename]{algorithm2e}
%
% -----------------------------------------------------------------------------
% PACKAGE listings
% -----------------------------------------------------------------------------
% These packages are required if you need to list code snippets.
\usepackage{listings}
% Nicely syntax highlighted m-code in LaTeX documents with stylefile mcode.sty
% http://www.mathworks.com/matlabcentral/fileexchange/8015-m-code-latex-package
\usepackage[numbered]{./tables_and_code/mcode}
%
% -----------------------------------------------------------------------------
% Re-define listings captions and titles based on language.
\newcaptionname{portuguese}{\lstlistingname}{Listagem} % Listings CAPTIONS
\newcaptionname{portuguese}{\lstlistlistingname}{Listagens} % LIST of LISTINGS
%
% -----------------------------------------------------------------------------
% PACKAGE csquotes
% -----------------------------------------------------------------------------
% Quotation helper package
\usepackage{csquotes}
%
% -----------------------------------------------------------------------------
% PACKAGE todonotes
% -----------------------------------------------------------------------------
% Create TODO Notes in text
% The notes can be made invisible by just using the 'disable' option:
\usepackage[textwidth=2cm, textsize=small]{todonotes}
%\usepackage[textwidth=2cm, textsize=small, disable]{todonotes}
\setlength{\marginparwidth}{2cm}
%
% -----------------------------------------------------------------------------
% PACKAGE changes
% -----------------------------------------------------------------------------
% Track changes in document (changes in pdf preview).
%% Use "final" option to make all tracking markups invisible.
%\usepackage[authormarkup=superscript,authormarkuptext=id,markup=underlined,ulem={ULforem,normalbf},final]{changes}
\usepackage[authormarkup=superscript,authormarkuptext=id,markup=underlined,ulem={ULforem,normalbf}]{changes}
% commands:
% \added[id=xx]{text}
% \deleted[id=xx]{text}
% \replaced[id=xx]{deleted text}{added text}
% -----------------------------------------------------------------------------
% PACKAGES xcolor, color
% -----------------------------------------------------------------------------
% These packages are required for list code snippets.
\usepackage{xcolor}
\usepackage{color}
% The following special color definitions are used in the IST Thesis
\definecolor{forestgreen}{RGB}{34,139,34}
\definecolor{orangered}{RGB}{239,134,64}
\definecolor{lightred}{rgb}{1,0.4,0.5}
\definecolor{orange}{rgb}{1,0.45,0.13}	
\definecolor{darkblue}{rgb}{0.0,0.0,0.6}
\definecolor{lightblue}{rgb}{0.1,0.57,0.7}
\definecolor{gray}{rgb}{0.4,0.4,0.4}
\definecolor{lightgray}{rgb}{0.95, 0.95, 0.95}
\definecolor{darkgray}{rgb}{0.4, 0.4, 0.4}
\definecolor{editorGray}{rgb}{0.95, 0.95, 0.95}
\definecolor{editorOcher}{rgb}{1, 0.5, 0} % #FF7F00 -> rgb(239, 169, 0)
\definecolor{chaptergrey}{rgb}{0.6,0.6,0.6}
\definecolor{editorGreen}{rgb}{0, 0.5, 0} % #007C00 -> rgb(0, 124, 0)
\definecolor{olive}{rgb}{0.17,0.59,0.20}
\definecolor{brown}{rgb}{0.69,0.31,0.31}
\definecolor{purple}{rgb}{0.38,0.18,0.81}
%
% -----------------------------------------------------------------------------
% PACKAGE setspace:
% ----------------------------------------------------------------------------
% Provides support for setting the spacing between lines in a document. 
% Package options include single spacing, one half spacing, and double spacing. 
% Alternatively the spacing can be changed as required with:
% \singlespacing, \onehalfspacing, and \doublespacing commands
\usepackage{setspace}
%
% -----------------------------------------------------------------------------
% PACKAGE paralist
% -----------------------------------------------------------------------------
% This package provides the 'inparaenum' environment for inline lists
\usepackage{paralist}
% usage:
% \begin{inparaenum}[(a)]
% \item bla
% \item bla, bla
% \end{inparaenum}
% -----------------------------------------------------------------------------
% PACKAGE cite:
% -----------------------------------------------------------------------------
% The 'cite' package will result in citation numbers being automatically
% sorted and properly "ranged". i.e.,
% [1], [2], [5]--[7], [9]
\usepackage{cite}
%
% -----------------------------------------------------------------------------
% PACKAGE acronym:
% -----------------------------------------------------------------------------
% The package 'acronym' garantees that all acronyms definitions are 
% given at the first usage. 
% IMPORTANT: do not use acronyms in titles/captions; otherwise the definition 
% will appear on the table of contents.
\usepackage[printonlyused]{acronym}
%
% -----------------------------------------------------------------------------
% PACKAGE hyperref
% -----------------------------------------------------------------------------
% Set links for references and citations in document
\usepackage{hyperref}
% pre-configuration of hyperref
\hypersetup{ colorlinks=true,
             citecolor=cyan,
             linkcolor=darkgray,
             urlcolor=teal,
             breaklinks=true,
             bookmarksnumbered=true,
             bookmarksopen=true,
             pdftitle=\@title, % THESIS TITLE
             pdfauthor=\@author,  % YOUR NAME
             pdfcreator=\@author,   % YOUR NAME
}
%
% -----------------------------------------------------------------------------
% PACKAGE url:
% -----------------------------------------------------------------------------
% Provides better support for handling and breaking URLs.
\usepackage{url} 
%
% -----------------------------------------------------------------------------
% PACKAGE Cleveref:
% -----------------------------------------------------------------------------
% Clever Referencing of document parts
% Note: portuguese is supported through "brazilian" option
\usepackage[\IfLanguageName{english}{english}{brazilian}]{cleveref}
%
% -----------------------------------------------------------------------------
% PACKAGE enumitem:
% -----------------------------------------------------------------------------
%For enhanced enumeration of lists
%\usepackage{enumitem}
\usepackage[shortlabels]{enumitem}
\setlist[description]{leftmargin=\parindent,labelindent=\parindent,itemsep=1pt,parsep=0pt,topsep=0pt}
%
% #############################################################################
% GLOBAL FORMATTING OF THE THESIS DOCUMENT before using FANCY stuff
% Set paragraph counter to alphanumeric mode
\renewcommand{\theparagraph}{\Alph{paragraph}~--}
\hoffset 0in
\voffset 0in
\oddsidemargin 0 cm
\evensidemargin 0 cm
\marginparsep 0in
\topmargin -0.25cm
\textwidth 16 cm
\textheight 22.4 cm
\makeatletter
% package indentfirst says \let\@afterindentfalse\@afterindenttrue
% and we revert this modification, reinstating the original definitio
% of \@afterindentfalse
\def\@afterindentfalse{\let\if@afterindent\iffalse}
\makeatother
% -----------------------------------------------------------------------------
% PACKAGE fancyhdr:
% -----------------------------------------------------------------------------
% The fancyhdr macro package allows to customize page headers and footers.
\usepackage{fancyhdr}
\pagestyle{fancy}
\renewcommand{\chaptermark}[1]{\markboth{\thechapter.\ #1}{}}
\renewcommand{\sectionmark}[1]{\markright{\thesection\ #1}}
\fancyhead{}
\renewcommand{\headrulewidth}{0.0pt}
\renewcommand{\footrulewidth}{0.0pt}
\addtolength{\headheight}{2pt} % make space for the rule
\fancypagestyle{plain}{%
   \fancyhead{} % get rid of headers
   \renewcommand{\headrulewidth}{0pt} % and the line
   \renewcommand{\footrulewidth}{0pt}
}
\fancypagestyle{blank}{%
   \fancyhf{} % get rid of headers and footers
   \renewcommand{\headrulewidth}{0pt} % and the line
   \renewcommand{\footrulewidth}{0pt}
}
\fancypagestyle{abstract}{%
   \fancyhead{}
   \renewcommand{\headrulewidth}{0pt}
   \renewcommand{\footrulewidth}{0.0pt}
}
\fancypagestyle{document}{%
	\fancyhead{}
	\renewcommand{\headrulewidth}{0.5pt}
	\renewcommand{\footrulewidth}{0.5pt}
	\addtolength{\headheight}{2pt} % make space for the rule
}
\setcounter{secnumdepth} {5}
\setcounter{tocdepth} {5}
\renewcommand{\thesubsubsection}{\thesubsection.\Alph{subsubsection}}
\renewcommand{\subfigtopskip}{0.3 cm}
\renewcommand{\subfigbottomskip}{0.2 cm}
\renewcommand{\subfigcapskip}{0.3 cm}
\renewcommand{\subfigcapmargin}{0.2 cm}
%
% -----------------------------------------------------------------------------
% PACKAGE minitoc:
% -----------------------------------------------------------------------------
% Package 'minitoc' creates a mini-table of contents (a “minitoc”) at 
% the beginning of each chapter of a document.
% This packages are required for the \fancychapter configuration
\usepackage{minitoc}
\setcounter{minitocdepth}{1}
\setlength{\mtcindent}{24pt}
\renewcommand{\mtcfont}{\small\rm}
\renewcommand{\mtcSfont}{\small\bf}
\renewcommand*{\kernafterminitoc}{\kern0.\baselineskip\kern0.ex}
\mtcselectlanguage{\languagename} 
% Now prepare the MINITOC
\def\boxedverbatim{%
  \def\verbatim@processline{%
    {\setbox0=\hbox{\the\verbatim@line}%
    \hsize=\wd0 \the\verbatim@line\par}}%
  \@minipagetrue%%%DPC%%%
  \@tempswatrue%%%DPC%%%
  \setbox0=\vbox\bgroup\vspace*{0.2cm}\footnotesize\verbatim
}
\def\endboxedverbatim{%
  \endverbatim
  \unskip\setbox0=\lastbox %%%DPC%%%
  \hspace*{0.2cm}
  \vspace*{-0.2cm}
  \egroup
  \fbox{\box0}% <<<=== change here for centering,...
}
% Now prepare the CHAPTER Number
\newcommand*{\chapnumfont}{%
%   \usefont{T1}{\@defaultcnfont}{b}{n}\fontsize{100}{130}\selectfont%
  \usefont{T1}{pbk}{b}{n}
  \fontsize{150}{130}
  \selectfont
  \color{chaptergrey}
}
\makeatletter
\def\@makechapterhead#1{%
  \vspace*{50\p@}%
  {\parindent \z@ \raggedright \normalfont
    {\chapnumfont\ifnum \c@secnumdepth >\m@ne
%         \huge\bfseries \@chapapp\space \thechapter
        \raggedleft\bfseries \thechapter
        \par\nobreak
        \vskip 20\p@
    \fi}
    \interlinepenalty\@M
    {\raggedleft\Huge \bfseries #1\par\nobreak}
    \vskip 40\p@
  }}
\makeatother
% Now put it all together as a command \fancychapter
\newcommand{\fancychapter}[1]{\chapter{#1}\vfill\minitoc\pagebreak}
%
% #############################################################################
% ADDITIONAL COMMANDS AND CONFIGURATIONS
% #############################################################################
% This commmand allows to place horizontal lines with a custom width... 
% replaces the standard hline command
\newcommand{\hlinew}[1]{%
  \noalign{\ifnum0=`}\fi\hrule \@height #1 \futurelet
   \reserved@a\@xhline}
%   
% -----------------------------------------------------------------------------
% This command defines some marks... USEFUL FOR TABLES.
\def\Mark#1{\raisebox{0pt}[0pt][0pt]{\textsuperscript{\footnotesize\ensuremath{\ifcase#1\or *\or \dagger\or \ddagger\or%
    \mathsection\or \mathparagraph\or \|\or **\or \dagger\dagger%
    \or \ddagger\ddagger \else\textsuperscript{\expandafter\romannumeral#1}\fi}}}}
%
% -----------------------------------------------------------------------------
% The following configurations are used for LISTINGS of certain languages
\lstdefinestyle{XML} {
	language=XML,
	extendedchars=true, 
	breaklines=true,
	breakatwhitespace=true,
	emph={},
	emphstyle=\color{red},
	basicstyle=\small,
	xleftmargin=17pt,
	columns=fullflexible,
	commentstyle=\color{gray}\upshape,
	morestring=[b][\color{brown}]",
	morecomment=[s]{<?}{?>},
	morecomment=[s][\color{forestgreen}]{<!--}{-->},
	keywordstyle=\color{orangered},
	stringstyle=\ttfamily\color{black},
	% stringstyle=\ttfamily\color{black}\normalfont,
	tagstyle=\color{blue},
	% tagstyle=\color{darkblue}\bf,
	morekeywords={asn,action,addrType,abilityNAT,audioSampleRate,audiChannels,,bandwidth,bitmapSize,bitRate,connection,codecs,concurrentLinks,dependency,duration,frameRate,from,height,ip,id,lang,mimeType,onlineTime,peerMode,port,priority,peerProtocol,property,release,to,tier,type,transactionID,url,uploadBWlevel,version,width},
	otherkeywords={attribute,xmlns,schemaLocation,PresentationType,availabilityStartTime,availabilityEndTime,minimumUpdatePeriod,minBufferTime,UpdateTime},
}
% ----------------------------------------------------------------------------
\lstdefinelanguage{Assembler}{
	morecomment=[l];,
	keywords={ADD,ADDC,SUB,SUBB,CMP,MUL,DIV,MOD,NEG,AND,OR,NOT,XOR,TEST,BIT,SET,EI,EI0,EI1,EI2,EI3,SETC,EDMA,CLR,DI,DI0,DI1,DI2,DI3,CLRC,SHR,SHL,SHRA,SHLA,ROR,ROL,RORC,ROLC,MOV,MOVB,MOVBS,MOVP,MOVL,MOVH,SWAP,PUSH,POP,JZ,JNZ,JN,JNN,JP,JNP,JC,JNC,JV,JNV,JEQ,JNE,JLT,JLE,JGT,JGE,JA,JAE,JB,JBE,JMP,CALL,CALLF,RET,RETF,SWE,RFE,NOP},
	morekeywords={EQU,TABLE,WORD,STRING,PLACE},
} 
% ----------------------------------------------------------------------------
\lstdefinestyle{coloredASM}{
	language=Assembler,
	extendedchars=false,
	breaklines=true,
	tabsize=2,
	numberstyle=\tiny,
	numbers=left,
	breakatwhitespace=true,
	emph={},
	emphstyle=\color{red},
	fontadjust=true,
	basicstyle=\small\ttfamily,
	% basicstyle=\footnotesize\ttfamily,
	columns=fixed,
	xleftmargin=17pt,
	framexleftmargin=17pt,
	framexrightmargin=5pt,
	framexbottommargin=4pt,
	commentstyle=\color{forestgreen}\upshape,
	morestring=[b][\color{brown}]",
	keywordstyle=\color{darkblue},
	stringstyle=\ttfamily\color{black},
	literate={á}{{\'a}}1 {ã}{{\~a}}1 {â}{{\^a}}1 {é}{{\'e}}1 {É}{{\'E}}1 {ê}{{\^e}}1 {õ}{{\~o}}1 {ó}{{\'o}}1 {í}{{\'i}}1 {ç}{{\c{c}}}1 {Ç}{{\c{C}}}1,
}    
% ----------------------------------------------------------------------------
\lstdefinelanguage{CSS}{
	sensitive=true,
	morecomment=[l]{//},
	morecomment=[s]{/*}{*/},
	morestring=[b]',
	morestring=[b]",
	alsoletter={:},
	alsodigit={-},
	keywords={color,background-image:,margin,padding,font,weight,display,position,top,left,right,bottom,list,style,border,size,white,space,min,width, transition:, transform:, transition-property, transition-duration, transition-timing-function}
}
% ----------------------------------------------------------------------------
% JavaScript
\lstdefinelanguage{JavaScript}{
	morecomment=[s]{/*}{*/},
	morecomment=[l]//,
	morestring=[b]",
	morestring=[b]',
	morekeywords={typeof, new, true, false, catch, function, return, null, catch, switch, var, if, in, while, do, else, case, break}
}
% ----------------------------------------------------------------------------
\lstdefinelanguage{HTML5}{
	language=html,
	sensitive=true,	
	alsoletter={<>=-},	
	morecomment=[s]{<!-}{-->},
	tag=[s],
	otherkeywords={
	% General
	>,
	% Standard tags
	<!DOCTYPE,
	</html, <html, <head, <title, </title, <style, </style, <link, </head, <meta, />,
	% body
	</body, <body,
	% Divs
	</div, <div, </div>, 
	% Paragraphs
	</p, <p, </p>,
	% scripts
	</script, <script,
	% More tags...
	<canvas, /canvas>, <svg, <rect, <animateTransform, </rect>, </svg>, <video, <source, <iframe, </iframe>, </video>, <image, </image>, <header, </header, <article, </article},
	ndkeywords={
	% General
	=,
	% HTML attributes
	charset=, src=, id=, width=, height=, style=, type=, rel=, href=,
	% SVG attributes
	fill=, attributeName=, begin=, dur=, from=, to=, poster=, controls=, x=, y=, repeatCount=, xlink:href=,
	% properties
	margin:, padding:, background-image:, border:, top:, left:, position:, width:, height:, margin-top:, margin-bottom:, font-size:, line-height:,
	% CSS3 properties
	transform:, -moz-transform:, -webkit-transform:,
	animation:, -webkit-animation:,
	transition:,  transition-duration:, transition-property:, transition-timing-function:,
	}
}
% ----------------------------------------------------------------------------
\lstdefinestyle{htmlcssjs} {%
	% General design
	backgroundcolor=\color{editorGray},
		fontadjust=true,
	basicstyle=\small\ttfamily,   
	frame=b,
	% line-numbers
	xleftmargin={0.75cm},
	numbers=left,
	stepnumber=1,
	firstnumber=1,
	numberfirstline=true,	
	% Code design
	identifierstyle=\color{black},
	keywordstyle=\color{blue}\bfseries,
	ndkeywordstyle=\color{editorGreen}\bfseries,
	stringstyle=\color{editorOcher}\ttfamily,
	commentstyle=\color{brown}\ttfamily,
	% Code
	language=HTML5,
	alsolanguage=JavaScript,
	alsodigit={.:;},	
	tabsize=2,
	showtabs=false,
	showspaces=false,
	showstringspaces=false,
	extendedchars=true,
	breaklines=true,
	% German umlauts
	literate=%
	{Ö}{{\"O}}1
	{Ä}{{\"A}}1
	{Ü}{{\"U}}1
	{ß}{{\ss}}1
	{ü}{{\"u}}1
	{ä}{{\"a}}1
	{ö}{{\"o}}1
}
% ----------------------------------------------------------------------------
\lstdefinestyle{py} {%
	language=python,
	literate=%
	*{0}{{{\color{lightred}0}}}1
	{1}{{{\color{lightred}1}}}1
	{2}{{{\color{lightred}2}}}1
	{3}{{{\color{lightred}3}}}1
	{4}{{{\color{lightred}4}}}1
	{5}{{{\color{lightred}5}}}1
	{6}{{{\color{lightred}6}}}1
	{7}{{{\color{lightred}7}}}1
	{8}{{{\color{lightred}8}}}1
	{9}{{{\color{lightred}9}}}1,
	basicstyle=\small\ttfamily,
	numbers=left,
	% numberstyle=\tiny,
	% stepnumber=2,
	numbersep=5pt,
	tabsize=4,
	extendedchars=true,
	breaklines=true,
	keywordstyle=\color{blue}\bfseries,
	frame=b,
	commentstyle=\color{brown}\itshape,
	stringstyle=\color{editorOcher}\ttfamily,
	showspaces=false,
	showtabs=false,
	xleftmargin=17pt,
	framexleftmargin=17pt,
	framexrightmargin=5pt,
	framexbottommargin=4pt,
	backgroundcolor=\color{lightgray},
	showstringspaces=false,
}

\colorlet{punct}{red!60!black}
\definecolor{background}{HTML}{EEEEEE}
\definecolor{delim}{RGB}{20,105,176}
\colorlet{numb}{magenta!60!black}

\lstdefinelanguage{json}{
    basicstyle=\normalfont\ttfamily,
    numbers=left,
    numberstyle=\scriptsize,
    stepnumber=1,
    numbersep=8pt,
    showstringspaces=false,
    breaklines=true,
    frame=lines,
    backgroundcolor=\color{background},
    literate=
     *{0}{{{\color{numb}0}}}{1}
      {1}{{{\color{numb}1}}}{1}
      {2}{{{\color{numb}2}}}{1}
      {3}{{{\color{numb}3}}}{1}
      {4}{{{\color{numb}4}}}{1}
      {5}{{{\color{numb}5}}}{1}
      {6}{{{\color{numb}6}}}{1}
      {7}{{{\color{numb}7}}}{1}
      {8}{{{\color{numb}8}}}{1}
      {9}{{{\color{numb}9}}}{1}
      {:}{{{\color{punct}{:}}}}{1}
      {,}{{{\color{punct}{,}}}}{1}
      {\{}{{{\color{delim}{\{}}}}{1}
      {\}}{{{\color{delim}{\}}}}}{1}
      {[}{{{\color{delim}{[}}}}{1}
      {]}{{{\color{delim}{]}}}}{1},
}
