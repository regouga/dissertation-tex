% #############################################################################
% RESUMO em Português
% !TEX root = ../main.tex
% #############################################################################
% use \noindent in firts paragraph
\noindent Os serviços de streaming de música são usados ​​diariamente por milhões de pessoas em todo o mundo, permitindo a escuta sob demanda e a descoberta de músicas, artistas e podcasts que se alinham estreitamente com as preferências do ouvinte. Por outro lado, as estações de rádio tradicionais persistem como um modo omnipresente e viável de escutar música e conteúdo pré-programado, incluindo notícias, relatórios de trânsito, e até informações meteorológicas. Embora ambos ofereçam aos ouvintes um conjunto distinto de funcionalidades, os esforços para combinar o 'melhor dos dois mundos' têm sido poucos. Com este objetivo em mente, descrevemos os nossos esforços para entender os hábitos dos consumidores de serviços de streaming de música e de estações de rádio tradicionais, com o objetivo final de desenvolver uma plataforma, nomeada Sterio, que cria uma experiência nova e integrada para ouvintes individuais e os seus círculos de amigos e família. Utilizando a prototipagem rápida e o método \textit{speed dating}, estudamos as implicações de design para a criação e validação de experiências auditivas semelhantes às das rádios tradicionais, que são pessoais, personalizáveis ​​e compartilháveis.