% #############################################################################
% Abstract Text
% !TEX root = ../main.tex
% #############################################################################
% use \noindent in firts paragraph
\noindent Audio streaming services are used daily by millions worldwide, enabling on-demand listening and the discovery of songs, artists and podcasts that closely align with the listener's preferences. Meanwhile, traditional terrestrial radio persists as another ubiquitous and still viable mode of accessing more pre-programmed music and news content, including traffic reports and weather information. While both media services offer listeners a distinct set of value propositions, efforts to combine the 'best of both worlds' have been few and far between. Towards this objective, we describe our preliminary efforts to understand audio media consumers' music streaming and traditional radio listening habits and preferences as part of a project aimed at creating an integrated experience for individual listeners and their close networks of family and friends. Through rapid prototyping, and the speed dating method, we explore the design implications for creating and validating radio-like experiences that are at once personal, customizable and shareable.